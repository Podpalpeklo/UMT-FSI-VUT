% !TeX root = skripta-konstitutivni-vztahy.tex
% !TeX lastmodified = 2018-12-10

\subsection{Misesova mezní podmínka energetická formulace}
Ekvivalentní napětí podle von Misese je v~kurzu PP na FSI v Brně odvozováno jako podmínka plasticity (HMH), založená na velikosti oktaedrického smykového napětí.
Ke stejné podmínce lze však dospět i~na základě energetického přístupu, kdy mezní stav (plasticity nebo pevnosti, podmínka je používána i~jako pevnostní kriterium) nastane tehdy, když měrná deformační energie změny tvaru (deviátorová) dosáhne jisté mezní hodnoty, která je materiálovou charakteristikou.
Ve velkých deformacích je ekvivalentní napětí skalární hodnotou Cauchyho tenzoru (skutečného) napětí.

V~jednoosé napjatosti je měrná deformační energie
\begin{equation}
	w_e = \frac{\diff W_e}{\diff V} = \int\limits_0^\varepsilon \sigma \diff \varepsilon
\end{equation}
a~pro lineárně elastický materiál se dá zjednodušit do tvaru
\begin{equation}
	w_e = \frac{1}{2} \sigma \varepsilon = \frac{\sigma^2}{2 E} = \frac{1}{2} E \varepsilon^2
\end{equation}

Pro víceosou napjatost je třeba do deformační energie zahrnout příspěvek všech složek tenzorů napětí a~přetvoření. Přitom složky s~nestejnými indexy spolu svírají pravý úhel, takže jejich příspěvek k~deformační práci je nulový.

Měrnou energii lze tedy zapsat ve tvaru:
\begin{equation}\begin{split}
	w_e
	= &\int\limits_{0}^{\varepsilon_{11}} \sigma_{11} \diff \varepsilon_{11}
	+ \int\limits_{0}^{\varepsilon_{12}} \sigma_{12} \diff \varepsilon_{12}
	+ \int\limits_{0}^{\varepsilon_{13}} \sigma_{13} \diff \varepsilon_{13}
	+ \int\limits_{0}^{\varepsilon_{21}} \sigma_{21} \diff \varepsilon_{21}\\
	+ &\int\limits_{0}^{\varepsilon_{22}} \sigma_{22} \diff \varepsilon_{22}
	+ \int\limits_{0}^{\varepsilon_{23}} \sigma_{23} \diff \varepsilon_{23}
	+ \int\limits_{0}^{\varepsilon_{31}} \sigma_{31} \diff \varepsilon_{31}
	+ \int\limits_{0}^{\varepsilon_{32}} \sigma_{32} \diff \varepsilon_{32}
	+ \int\limits_{0}^{\varepsilon_{33}} \sigma_{33} \diff \varepsilon_{33}
\end{split}\end{equation}

Tenzorovým zápisem lze zjednodušit do tvaru
\begin{equation}
	w_e
	= \int\limits_{0}^{\bm{\varepsilon}} \bm{\sigma} \!:\! \bm{\varepsilon} 
	\:\Leftrightarrow\: \int\limits_{0}^{\varepsilon_{ij}} \sigma_{ij} \diff \varepsilon_{ij},
\end{equation}
kde $\bm{\sigma}$ a~$\bm{\varepsilon}$ jsou tenzory druhého řádu a~v~indexovém zápisu platí Einsteinovo sčítací pravidlo.

Použijeme-li ve vztahu rozklad tenzorů napětí a~deformace na jejich kulové a~deviátorové složky, dostaneme měrnou energii napjatosti následovně:
\begin{equation}\begin{split}%todo už nepoužíváme index e?
	w
	= &\int\limits_0^{\bm{\varepsilon}} \left(\bm{\sigma}' + \frac{\mathrm{Sp}(\bm{\sigma})}{3}\bm{1}\right)
	\!:\! \left(\diff \bm{\varepsilon}' + \frac{\mathrm{Sp}(\diff \bm{\varepsilon})}{3}\bm{1}\right)\\
	= &\int\limits_0^{\bm{\varepsilon}'} \bm{\sigma}' \!:\! \diff \bm{\varepsilon}'
	+ \int\limits_0^{\bm{\varepsilon}} \left( \frac{\mathrm{Sp}(\bm{\sigma})}{3}\bm{1} \right) \!:\! \left( \frac{\mathrm{Sp}(\diff \bm{\varepsilon})}{3}\bm{1} \right)
	= w_d + w_V,
\end{split}\end{equation}
kde čárka v~horním indexu značí deviátorovou složku tenzoru.

Měrná energie napjatosti tedy sestává z~části deviátorové $w_d$ (energie změny tvaru) a~objemové $w_V$ (energie změny objemu).

Misesova podmínka vychází z deviátorové části energie napjatosti $w_d$ (dané změnou tvaru), kterou lze tedy zapsat následovně:
\begin{equation}
	w_d = \int\limits_0^{\bm{\varepsilon}'} \bm{\sigma}' \!:\! \diff \bm{\varepsilon}' \:\Leftrightarrow\: \int\limits_0^{\varepsilon'_{ij}} \sigma'_{ij} \diff \varepsilon'_{ij}
\end{equation}

Použijeme-li pro lineárně elastický materiál Hookeův zákon s~oddělenou tvarovou a~objemovou složkou, platí:
\begin{equation}
	\sigma_{ij}
	= 2 G \varepsilon_{ij} + \lambda \varepsilon_{ii}
	= 2 G \varepsilon'_{ij} + K \varepsilon_{ii}
\end{equation}
a~tedy pro deviátorové složky napětí a~přetvoření
\begin{equation}
	\sigma'_{ij} = 2 G \varepsilon'_{ij}
\end{equation}
nebo inverzní tvar pro hodnotu nebo přírůstek deviátoru deformace
\begin{equation}
	\varepsilon'_{ij} = \frac{\sigma'_{ij}}{2 G} \qquad
	\diff\varepsilon'_{ij} = \frac{\diff \sigma'_{ij}}{2 G}
\end{equation}
Pak lze měrnou energii tvarové změny vyjádřit pomocí pouze napětí anebo pomocí přetvoření následovně:
\begin{equation}
	w_d
	= \int\limits_0^{\varepsilon'_{ij}} \sigma'_{ij} \diff \varepsilon'_{ij}
	= \frac{1}{2} \sigma'_{ij} \varepsilon'_{ij}
	= \frac{1}{4 G} \sigma'_{ij} \sigma'_{ij}
	= G \varepsilon'_{ij} \varepsilon'_{ij}
	= G \bm{\varepsilon}' \!:\! \bm{\varepsilon}'
\end{equation}

Zavede-li se redukované napětí vztahem
\begin{equation}
	\sigma_\text{ekv}^2
	= \frac{3}{2} (\bm{\sigma}' \!:\! \bm{\sigma}')
	\:\Leftrightarrow\: \sigma_{ij}' \sigma_{ij}',
\end{equation}
pak lze deviátorovou část energie napjatosti $w_d$ zapsat následovně:
\begin{equation}
	w_d = \frac{\sigma_\text{ekv}^2}{6 G}.
\end{equation}

Zavede-li se podobně redukované přetvoření podle Misesovy podmínky, lze energii vyjádřit také s~jeho použitím
\begin{equation}
	w_d
	= \frac{2}{3} G (1+\mu)^2 \varepsilon_\text{ekv}^2
	= \frac{E (1+\mu)}{3} \varepsilon_\text{ekv}^2.
\end{equation}

\subsubsection{Misesova mezní podmínka jako izotropní zjednodušení podmínky Tsai-Hill pro kompozity}
Misesova energetická podmínka je obecně používána i~jako podmínka pevnosti u~technických i~biologických materiálů.
Nejsnadněji lze shodu obou přístupů ukázat pomocí energetické podmínky Tsai-Hill uváděné\footnote{Agarval, Broutman: Vláknové kompozity. SNTL Praha, 1987} pro vláknové kompozity ve tvaru
\begin{equation}
	\left(\frac{\sigma_L}{\sigma_{PL}}\right)^2
	- \frac{\sigma_L}{\sigma_{PL}} \frac{\sigma_T}{\sigma_{PT}}
	+ \left(\frac{\sigma_T}{\sigma_{PT}}\right)^2
	+ \left(\frac{\tau_{LT}}{\tau_P}\right)^2 < 1,
\end{equation}
kde indexy $L$, $T$ označují hlavní směry materiálu a~index $P$ pevnost v~daném směru.

Po zjednodušení na materiál izotropní ($\sigma_{PL}= \sigma_{PT} = \sigma_P$) a~zavedení pevnosti ve smyku podle Misesovy podmínky ve tvaru
\begin{equation}
	\tau_P = \frac{\sigma_P}{\sqrt{3}},
\end{equation}
převedeme výše uvedenou nerovnici na tvar
\begin{equation}
	\left(\frac{\sigma_L}{\sigma_{PL}}\right)^2
	- \frac{\sigma_L \sigma_P}{\sigma_P^2}
	+ \left(\frac{\sigma_T}{\sigma_P}\right)^2
	+ \left(\frac{\sqrt{3} \tau_{LT}}{\sigma_P}\right)^2 < 1.
\end{equation}

Výsledný tvar podmínky (platný pro rovinnou napjatost) pak je
\begin{equation}\label{tsai-hill-kompozit}
	\sqrt{\sigma_L^2 - \sigma_L\sigma_T + \sigma_T^2 + 3\tau_{LT}^2} < \sigma_P.
\end{equation}

Redukované (ekvivalentní) napětí podle Misesovy podmínky (HMH) se pro obecný souřadnicový systém obvykle uvádí ve tvaru\footnote{Janíček, Ondráček Vrbka, Burša: Mechanika těles -- Pružnost a pevnost I. CERM Brno, 2006.}
\begin{equation}
	\sigma_\text{red}
	= \sqrt{\frac{1}{2} \bigg[ (\sigma_L-\sigma_T)^2 + (\sigma_L-\sigma_{T'})^2 + (\sigma_T-\sigma_{T'})^2 + 6 \left( \tau_{LT}^2 + \tau_{LT'}^2 + \tau_{TT'}^2 \right) \bigg]},
\end{equation}
kde označení os $x$, $y$, $z$ bylo pouze změněno na $L$, $T$, $T'$.
Pro dvouosou napjatost v~rovině $LT$ pak dostáváme tvar
\begin{equation}
	\sigma_\text{red} = \sqrt{\sigma_L^2 - \sigma_L\sigma_T + \sigma_T^2 + 3\tau_{LT}^2},
\end{equation}
z~nějž porovnáním s~rovnicí (\ref{tsai-hill-kompozit}) plyne obvyklý tvar mezní podmínky
\begin{equation}
	\sigma_\text{red} < \sigma_P,
\end{equation}
který se od obvyklé Misesovy podmínky plasticity liší jen použitou mezní hodnotou.
