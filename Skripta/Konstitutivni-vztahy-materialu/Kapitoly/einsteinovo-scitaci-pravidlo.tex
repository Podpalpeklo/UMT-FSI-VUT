% !TeX root = skripta-konstitutivni-vztahy.tex
% !TeX lastmodified = 2006-11-04

\subsection{Einsteinovo sčítací pravidlo}\label{sec:einsteinovo-scitaci-pravidlo}
Einsteinovo sčítací pravidlo se používá pro zjednodušení obecných tenzorových zápisů. Vyskytuje-li se v~některém členu opakovaný index (tzv. sčítací index, v~našem případě index~$k$) pak se provádí sumace přes tento index. Např. obecný zápis \hyperref[sec:green-lagrange]{Green-Lagrangeova} tenzoru přetvoření ve tvaru
\begin{equation}
	E^L_{ij} = \frac{1}{2} \left[ \frac{\partial u_i}{X_j} + \frac{\partial u_j}{X_i} + \frac{\partial u_k}{X_j} \frac{\partial u_k}{X_i} \right]
\end{equation}
je třeba interpretovat následovně:
\begin{equation}
	E^L_{ij} = \frac{1}{2} \left[ \frac{\partial u_i}{X_j} + \frac{\partial u_j}{X_i} + \sum_{k=1}^3 \frac{\partial u_k}{X_j} \frac{\partial u_k}{X_i} \right]
\end{equation}
