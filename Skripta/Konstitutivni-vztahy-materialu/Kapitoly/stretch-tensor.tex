% !TeX root = skripta-konstitutivni-vztahy.tex
% !TeX lastmodified = 2014-11-13

\subsection{Tenzor protažení (stretch tensor)}
I~v~nedeformovaném stavu elementu může mít matice definující tenzor deformačního gradientu $\bm{F}$ složky odlišné od jednotkové matice; je to dáno případnou rotací elementu v~důsledku velkých deformací tělesa.  Polární dekompozicí tenzoru $\bm{F}$ lze tento rozložit na tenzor rotace $\bm{R}$ (vyjadřující rotaci tuhého tělesa) a~tenzor protažení $\bm{U}$ (popisující deformaci tělesa). Tento tenzor již je symetrický a~energeticky konjugovaný se symetrickou částí $\bm{T_B}$.

Tenzor deformačního gradientu je definován vztahem
\begin{equation}
	F_{ij} = \frac{\partial x_i}{\partial X_j}
\end{equation}
který můžeme přepsat do maticové podoby s~konečnými diferencemi namísto derivací
\begin{equation}
	\bm{F} = \frac{\Delta \bm{x}}{\Delta \bm{X}}
\end{equation}

Polární dekompozice tenzoru $\bm{F}$ pak spočívá v~jeho rozložení na tenzor 
rotace $\bm{R}$ a~tenzor protažení (stretch tensor) $\bm{U}$:
\begin{equation}
	\bm{F} = \bm{R} \bm{U}
\end{equation}

Tenzor rotace $\bm{R}$ je v rovině (2D zjednodušení) definován vztahem
\begin{equation}
	\bm{R} = \left(\begin{matrix}
		\cos(\psi) & \sin(\psi)\\
		-\sin(\psi) & \cos(\psi)\\
	\end{matrix}\right),
\end{equation}
kde úhel $\psi$ lze určit ze složek tenzoru deformačního gradientu pomocí vztahu
\begin{equation}
	\psi = \arctan\left(\frac{F_{12} - F_{21}}{F_{11} + F_{22}}\right)
\end{equation}

Pak lze tenzor protažení $\bm{U}$ určit ze vztahu
\begin{equation}
	\bm{U} = \bm{R}^{-1} \bm{F} = \bm{R}^T \bm{F}
\end{equation}

využívajícího ortogonality tenzoru rotace $\bm{R}$, pro který tedy inverzní matice se rovná matici transponované. Hlavní složky tenzoru $\bm{U}$ jsou hlavní poměrná protažení; tento tenzor je vždy symetrický na rozdíl od tenzoru deformačního gradientu $\bm{F}$.
