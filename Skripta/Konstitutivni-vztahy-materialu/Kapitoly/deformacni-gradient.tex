% !TeX root = skripta-konstitutivni-vztahy-materialu.tex
% !TeX lastmodified = 2019-04-10

\subsection{Tenzor deformačního gradientu}
Složkami tenzoru deformačního gradientu $\bm{F}$ v~hlavním souřadnicovém systému jsou poměrná protažení
\begin{equation}
	\lambda_x  = \frac{\partial x}{\partial X}
	\qquad
	\lambda_y  = \frac{\partial y}{\partial Y}
	\qquad
	\lambda_z  = \frac{\partial z}{\partial Z}
\end{equation}
Obecně je lze zapsat ve tvaru
\begin{equation}
	\lambda_{ij} = \frac{\partial x_i}{\partial X_j}
\end{equation}

Úplný maticový zápis v~obecném souřadnicovém systému je
\begin{equation}
	\bm{F} = \frac{\partial x_i}{\partial X_j} = \left(\begin{matrix}
		\frac{\partial x_1}{\partial X_1} & \frac{\partial x_1}{\partial X_2} & \frac{\partial x_1}{\partial X_3}\\
		\frac{\partial x_2}{\partial X_1} & \frac{\partial x_2}{\partial X_2} & \frac{\partial x_2}{\partial X_3}\\
		\frac{\partial x_3}{\partial X_1} & \frac{\partial x_3}{\partial X_2} & \frac{\partial x_3}{\partial X_3}
	\end{matrix}\right)
\end{equation}

Třetí invariant tenzoru deformačního gradientu je dán determinantem této matice $\bm{F}$, který lze nejsnáze určit z~hlavních hodnot poměrných protažení pomocí vztahu
\begin{equation}
	J = \det(\bm{F}) = \lambda_1 \lambda_2 \lambda_3
\end{equation}

\subsubsection{Vlastnosti tenzoru deformačního gradientu}
Tenzor deformačního gradientu je obecně nesymetrický, v~důsledku toho existují dva v~principu odlišné Cauchy-Greenovy tenzory deformace (levý a~pravý).

Třetí invariant $J$ tenzoru deformačního gradientu $\bm{F}$ udává poměrnou objemovou změnu elementu, jak plyne z následujícího vztahu: 
\begin{equation}
	e
	= \frac{V_\text{def} - V_\text{nedef}}{V_\text{nedef}}
	= \frac{\diff x \diff y \diff z - \diff X \diff Y \diff Z}{\diff X \diff Y \diff Z}
	= \frac{\partial x}{\partial X} \frac{\partial y}{\partial Y} \frac{\partial z}{\partial Z} - 1
	= \lambda_1 \lambda_2 \lambda_3 -1
	= J - 1
\end{equation}

Jako každý tenzor lze i~tenzor deformačního gradientu tedy rozložit na část kulovou (změna objemu) a~deviátorovou (změna tvaru). 

U~smluvních přetvoření byla kulová část tenzoru dána aritmetickým průměrem hlavních přetvoření. Podobně zde je kulová část dána průměrem hlavních souřadnic tenzoru, ovšem nikoli aritmetickým, nýbrž geometrickým, protože změna objemu je dána součinem těchto souřadnic.

Pro střední protažení platí tedy
\begin{equation}
	\lambda_s
	= \sqrt[3]{\lambda_1 \lambda_2 \lambda_3}
	= \sqrt[3]{J}
	= J^{\frac{1}{3}}
\end{equation}



