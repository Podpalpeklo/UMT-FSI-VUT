% !TeX root = skripta-konstitutivni-vztahy.tex
% !TeX lastmodified = 2018-12-10

\subsection{Doporučená literatura}
\begin{itemize}
	\item J. Hůlka: Výpočtová predikce tvárného porušování. Disertační práce VUT Brno, 2014.
	\item Kubík P.: Implementace, kalibrace a využití podmínek tvárného lomu v programech MKP. Disertační práce VUT Brno, 2015.
	\item články v časopisech uvedené u jednotlivých modelů.
\end{itemize}

Plastické (tvárné) materiály vykazují kromě malých elastických přetvoření konečná (tj. velká) nevratná přetvoření. V~jejich důsledku je nutné pro část tahové křivky mezi $R_e$ a~$R_m$ přepočítat smluvní hodnoty napětí a~přetvoření na \enquote{skutečné napětí -- logaritmické přetvoření}.

Nad hodnotou $R_m$ se vytvoří krček a~v~něm vzniká víceosá napjatost, kdy je třeba použít redukované napětí a~přetvoření a~jednoduchý přepočet není možný (vliv geometrie krčku, tedy jeho průměru a~poloměru zaškrcení, na pole napjatosti).

\subsection{Popis plastické deformace}
Nejvíce rozšířená teorie předpokládá nestlačitelný materiál, takže popisuje pouze deviátorovou složku elastických a~plastických deformací.
K~popisu pružně-plastického chování materiálu je nutné znát:
\begin{enumerate}
	\item Mezní podmínku plasticity
	\begin{itemize}
		\item Tresca
		\item von Mises
		\item \hyperref[sec:mohr-coulomb]{Mohr-Coulomb}
		\item \hyperref[sec:drucker-prager]{Drucker-Prager}
	\end{itemize}
	\item Model plastického tečení -- křivka zpevnění pro jednoosou napjatost, příp. proporcionální zatěžování (flow curve)
	\begin{itemize}
		\item \hyperref[sec:ramberg-osgood]{Ramberg-Osgood}
		\item \hyperref[sec:model-voce-4]{Voce 4}
		\item \hyperref[sec:norton]{Norton} -- model viskoplastického creepu
		\item \hyperref[sec:johnson-cook-viskoplasticita]{Johnson-Cook} -- viskoplastický model s~vlivem teploty
	\end{itemize}
	\item Model zpevnění materiálu
	\begin{itemize}
		\item Izotropní (nevhodné pro cyklické zatěžování)
		\item Kinematické (popisuje Bauschingerův efekt) -- model \hyperref[sec:chaboche]{Chaboche}
	\end{itemize}
	\item Model plastického porušení
	\begin{itemize}
		\item \hyperref[sec:johnson-cook-poruseni]{Johnson-Cook}
		\item \hyperref[sec:bai-wierzbicki]{Bai-Wierzbicki}
	\end{itemize}
\end{enumerate}

\subsubsection{Modely plastického porušení}
\begin{enumerate}
	\item Svázané modely plasticity -- zahrnují porušení do modelu plastického tečení, resp. popisují plastické chování za mezí pevnosti na základě popisu průběhu porušování (continuum damage mechanics -- CDM).
	\begin{itemize}
		\item Gurson-Tvergaard-Needleman -- model\footnote{A. Gurson, “Continuum theory of ductile rupture by void nucleation and growth. Part I. Yield criteria and flow rules for porous ductile media,” no. September, 1975.}\footnote{V. Tvergaard and A. Needleman, “Analysis of the cup-cone fracture in a round tensile bar,” Acta Metall., vol. 32, no. I, pp. 157–169, 1984.}\footnote{K. Nahshon and J. W. Hutchinson, “Modification of the Gurson Model for shear failure,” Eur. J. Mech. - A/Solids, vol. 27, no. 1, pp. 1–17, Jan. 2008.} pro porézní kovy
	\end{itemize}
	\item Nesvázané -- modelují pouze porušení
	\begin{itemize}
		\item \hyperref[sec:johnson-cook-poruseni]{Johnson-Cook} -- závislost lomového přetvoření na rychlosti deformace a~teplotě.
		\item \hyperref[sec:bai-wierzbicki]{Bai-Wierzbicki}
	\end{itemize}
\end{enumerate}
