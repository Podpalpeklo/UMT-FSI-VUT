% !TeX root = skripta-konstitutivni-vztahy.tex
% !TeX lastmodified = 2018-11-13

\subsection{Energeticky konjugované tenzory}
Pro správné (jednoznačné) určení energie napjatosti je nutné pracovat se vzájemně si odpovídajícími tenzory napětí a~přetvoření.
Těmto dvojicím tenzorů říkáme energeticky konjugované tenzory.
Jsou to tedy vzájemně přiřazené dvojice tenzorů napětí a~přetvoření, jejichž vzájemnou kombinací lze dostat (i~v~případě velkých přetvoření a~velkých posuvů) energii napjatosti.

Takto konjugované jsou např.
\begin{itemize}
	\item Green-Lagrangeův tenzor přetvoření a~2.~Piola-Kirchhoffův tenzor napětí,
	\item pravý Cauchy-Greenův tenzor deformace a~2.~Piola-Kirchhoffův tenzor napětí,
	\item tenzor protažení $\bm{U}$ (pravý) a~Biotův tenzor napětí $\bm{T}_B$ (symetrický, v~těchto oporách není popsán),
	\item tenzor deformace daný vztahem $\bm{F} - \bm{1}$ (jednotkový tenzor) a~1.~Piola-Kirchhoffův tenzor napětí.
\end{itemize}
