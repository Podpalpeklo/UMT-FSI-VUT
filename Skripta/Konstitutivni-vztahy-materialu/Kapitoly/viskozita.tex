% !TeX root = skripta-konstitutivni-vztahy.tex
% !TeX lastmodified = 2017-09-19

\section{Co je to viskozita?}
Viskozita je vlastnost kapaliny (tekutiny), charakterizující její odpor proti tvarovým změnám.
Je definována v~hydromechanice dvěma způsoby:

\textbf{Dynamická viskozita} $\eta$, jako konstanta úměrnosti mezi gradientem rychlosti v~kapalině a~odpovídajícím smykovým napětím v~ní, a~to v~podobě Newtonova zákona viskozity
\begin{equation}
	\tau = \eta \frac{\diff c}{\diff n},
\end{equation}
kde
\begin{description}
	\item[{$\tau\:[\si{\pascal}]$}] je smykové napětí,
	\item[{$\eta\:[\si{\pascal\second}]$}] je tzv. dynamická viskozita,
	\item[{$\tfrac{\diff c}{\diff n}\:[\si{\per\second}]$}] je gradient rychlosti kapaliny.
\end{description}

Kromě dynamické viskozity se v~hydromechanice definuje i~tzv. \textbf{kinematická viskozita} $\nu\:[\si{\meter\squared\per\second}]$ a~to vztahem
\begin{equation}
	\nu = \frac{\eta}{\varrho},
\end{equation}
kde
\begin{description}
	\item[{$\varrho\:[\si{\kilogram\per\meter\cubed}]$}] je hustota (měrná hmotnost) kapaliny.
\end{description}
