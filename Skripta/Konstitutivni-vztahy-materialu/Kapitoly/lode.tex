% !TeX root = skripta-konstitutivni-vztahy.tex
% !TeX lastmodified = 2018-12-11

\subsection{Lode}
\subsubsection{Lodeho parametr}
Dalším významným parametrem pro tvárné porušení vztaženým ke stavu napjatosti je Lodeho parametr $\mu$:
\begin{equation}
	\mu = \frac{2 \sigma_2 - \sigma_1 - \sigma_3}{\sigma_1 - \sigma_3}
\end{equation}

Lodeho parametr nabývá hodnot v~intervalu $-1 \leq \mu \leq 1$ a~charakterizuje typ napjatosti z~hlediska polohy hlavního napětí $\sigma_2$ vůči $\sigma_1$ a~$\sigma_3$. Jeho význam se různí podle materiálů, je velký např. u~hliníkových slitin, u~ocelí pak při nízkých faktorech triaxiality napětí.
\begin{figure}[H]\centering\begin{tabular}{lll}
	\toprule
	Název            &       $\mu$        &     Hlavní napětí      \\ \midrule
	Smyková           &         $0$          &     $\sigma_3= -\sigma_1; \sigma_2=0$      \\
	Jednoosá tahová       &         $-1$         &     $\sigma_1>0; \sigma_2= \sigma_3=0$     \\
	Dvouosá tahová rovnoměrná  &         $1$          &     $\sigma_1=\sigma_2>0; \sigma_3=0$      \\
	Jednoosá tlaková      &         $1$          &     $\sigma_3<0; \sigma_2= \sigma_1=0$     \\
	Dvouosá tlaková rovnoměrná &         $-1$         &     $\sigma_3=\sigma_2<0; \sigma_1=0$      \\
	Trojosá rovnoměrná     & Neurčitý výraz $0/0$ & $\sigma_1=\sigma_2=\sigma_3>0$\\
\bottomrule
\end{tabular}\end{figure}

\subsubsection{Lodeho úhel}
Alternativním vyjádřením Lodeho parametru je Lodeho úhel $\theta$, který představuje normalizovaný arkustangens Lodeho parametru $\mu$; lze jej také vyjádřit pomocí normalizovaného třetího invariantu deviátoru napětí $\xi$:
\begin{equation}
	\theta = \arctan\left(\frac{\mu}{\sqrt{3}}\right)
\end{equation}
nebo
\begin{equation}
	\theta = -\frac{1}{3}\arcsin\left(\xi\right),
\end{equation}
kde
\begin{equation*}
	\xi = \left(\frac{r}{\sigma_\text{Mises}}\right)^3
	= \frac{27}{2} \frac{\det \bm{S}}{\sigma_\text{Mises}^3},
\end{equation*}
přičemž
\begin{equation*}
	r = \sqrt[3]{\frac{9}{2} \bm{S}\cdot\bm{S}\!:\!\bm{S}}
	= \sqrt[3]{\frac{27}{2} \det\bm{S}}
	= \sqrt[3]{\frac{27}{2} (\sigma_1-\sigma_m)(\sigma_2-\sigma_m)(\sigma_3-\sigma_m)},
\end{equation*}
kde
\begin{description}
	\item[$\bm{S}$] reprezentuje deviátor tenzoru napětí,
	\item[$\sigma_\text{Mises}$] je střední (hydrostatická) složka napětí,
	\item[$\sigma_m$] je redukované napětí podle von Misesovy podmínky plasticity.
\end{description}

Lodeho úhel nabývá hodnot v~intervalu $-\frac{\pi}{6} \leq \theta \leq \frac{\pi}{6}$, normalizovaný třetí invariant deviátoru napětí je v~intervalu $-1 \leq \xi \leq 1$. 

Především v mezních podmínkách porušení má faktor triaxiality napětí a~Lodeho parametr velký význam, protože ovlivňují vznik porušení. Experimentální ověřování mezních podmínek je třeba proto provádět při různých hodnotách těchto parametrů; lze je měnit typem namáhání vzorku (tah, tlak, biaxiální tah-tlak, krut, různé jejich kombinace) nebo změnou jeho geometrie (poloměru vrubu).