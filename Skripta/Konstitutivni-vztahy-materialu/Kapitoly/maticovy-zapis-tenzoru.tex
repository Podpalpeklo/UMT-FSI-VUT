% !TeX root = ../skripta-konstitutivni-vztahy-materialu.tex

\section{Zkrácený zápis tenzorů}
Složky tenzoru v~prostoru $\mathbb{R}^3$ lze zkráceně zapsat v~prostoru $\mathbb{R}^6$. Motivací je snadný zápis konstitutivních vztahů a~menší náročnost na paměť a~počet výpočtů počítače.

Symetrický tenzor druhého řádu $3 \times 3$, jehož složky mají indexy $i,j \in \{1,2,3\}$ lze zapsat jako vektor s~indexy $a \in \{1,2,3,4,5,6\}$
\begin{equation*}
	T_{ij} =
	\begin{pmatrix}
		T_{11} & T_{12} & T_{13} \\ & T_{22} & T_{23} \\ \text{\small sym.} & & T_{33} \\
	\end{pmatrix}
	\quad\rightarrow\quad
	T_a =
	\begin{pmatrix}
		T_1 \\ T_2 \\ T_3 \\ T_4 \\ T_5 \\ T_6 \\
	\end{pmatrix}
\end{equation*}
a~symetrický tenzor čtvrtého řádu $3 \times 3 \times 3 \times 3$, jehož složky mají indexy $i,j,k,l \in \{1,2,3\}$ lze zapsat jako matici $6 \times 6$ s~indexy $a,b \in \{1,2,3,4,5,6\}$
\begin{equation*}
	T_{ijkl}
	\quad\rightarrow\quad
	T_{ab} =
	\begin{pmatrix}
		T_{11} & T_{12} & T_{13} & T_{14} & T_{15} & T_{16} \\
		& T_{22} & T_{23} & T_{24} & T_{25} & T_{26} \\
		& & T_{33} & T_{34} & T_{35} & T_{36} \\
		& & & T_{44} & T_{45} & T_{46} \\
		& \text{\small sym.} & & & T_{55} & T_{56} \\
		& & & & & T_{66} \\
	\end{pmatrix}
\end{equation*}

V~mechanice je potřeba při zápisu tenzoru napětí a~tenzoru přetvoření pomocí vektorů zachovat hodnotu hustoty elastické energie, která je produktem $2E = \bm{\sigma}\kontr\bm{\varepsilon} = \{\sigma\}^T\cdot\{\varepsilon\}$. Tradiční Voigtův\footnote{Woldemar \textsc{Voigt} (Němec, 1850---1919)} zápis i~méně používaný Kelvinův\footnote{William \textsc{Thomson} -- lord Kelvin (Skot, 1824---1907)}-Mandelův\footnote{Jean \textsc{Mandel} (Francouz, 1907-1982)} zápis toho dosahují různými mapováními.

\subsection{Voigtův zápis}
Tenzor napětí
\begin{equation*}
	\sigma_{ij} =
	\begin{pmatrix}
		\sigma_{11} & \sigma_{12} & \sigma_{13} \\ & \sigma_{22} & \sigma_{23} \\ \text{sym.} & & \sigma_{33} \\
	\end{pmatrix} =
	\begin{pmatrix}
		\sigma_{11} & \tau_{12} & \tau_{13} \\ & \sigma_{22} & \tau_{23} \\ \text{sym.} & & \sigma_{33} \\
	\end{pmatrix}
	\quad\rightarrow\quad
	\sigma_a =
	\begin{pmatrix}
		\sigma_{11} \\ \sigma_{22} \\ \sigma_{33} \\ \sigma_{23} \\ \sigma_{13} \\ \sigma_{12} \\
	\end{pmatrix} =
	\begin{pmatrix}
		\sigma_{11} \\ \sigma_{22} \\ \sigma_{33} \\ \tau_{23} \\ \tau_{13} \\ \tau_{12} \\
	\end{pmatrix}
\end{equation*}

Tenzor přetvoření
\begin{equation*}
	\varepsilon_{ij} =
	\begin{pmatrix}
		\varepsilon_{11} & \varepsilon_{12} & \varepsilon_{13} \\ & \varepsilon_{22} & \varepsilon_{23} \\ \text{\small sym.} & & \varepsilon_{33} \\
	\end{pmatrix} =
	\begin{pmatrix}
		\varepsilon_{11} & \frac{\gamma_{12}}{2} & \frac{\gamma_{13}}{2} \\ & \varepsilon_{22} & \frac{\gamma_{23}}{2} \\ \text{\small sym.} & & \varepsilon_{33} \\
	\end{pmatrix}
	\quad\rightarrow\quad
	\varepsilon_{a} =
	\begin{pmatrix}
		\varepsilon_{11} \\ \varepsilon_{22} \\ \varepsilon_{33} \\ 2 \varepsilon_{23} \\ 2 \varepsilon_{13} \\ 2 \varepsilon_{12} \\
	\end{pmatrix} =
	\begin{pmatrix}
		\varepsilon_{11} \\ \varepsilon_{22} \\ \varepsilon_{33} \\ \gamma_{23} \\ \gamma_{13} \\ \gamma_{12} \\
	\end{pmatrix}
\end{equation*}

Tenzor tuhosti
\begin{equation*}
	C_{ijkl}
	\quad\rightarrow\quad
	C_{ab} =
	\begin{pmatrix}
	C_{1111} & C_{1122} & C_{1133} & C_{1123} & C_{1113} & C_{1112} \\
	         & C_{2222} & C_{2233} & C_{2223} & C_{2213} & C_{2212} \\
	         &          & C_{3333} & C_{3323} & C_{3313} & C_{3312} \\
	         &          &          & C_{2323} & C_{2313} & C_{2312} \\
	      & \text{\small sym.} &          &          & C_{1313} & C_{1112} \\
	         &          &          &          &          & C_{1212} \\
	\end{pmatrix}
\end{equation*}

Výhody:
\begin{itemize}
	\item zachování hustoty elastické energie
	\item zachování složek tenzoru tuhosti
\end{itemize}

Nevýhody:
\begin{itemize}
	\item napětí a~přetvoření je mapováno různými způsoby
	\item norma tenzorů se nezachovává
\end{itemize}

\subsection{Kelvinův-Mandelův zápis}
Tenzor napětí
\begin{equation*}
\sigma_{ij} =
\begin{pmatrix}
\sigma_{11} & \sigma_{12} & \sigma_{13} \\ & \sigma_{22} & \sigma_{23} \\ \text{sym.} & & \sigma_{33} \\
\end{pmatrix} =
\begin{pmatrix}
\sigma_{11} & \tau_{12} & \tau_{13} \\ & \sigma_{22} & \tau_{23} \\ \text{sym.} & & \sigma_{33} \\
\end{pmatrix}
\quad\rightarrow\quad
\sigma_a =
\begin{pmatrix}
\sigma_{11} \\ \sigma_{22} \\ \sigma_{33} \\ \sqrt{2} \sigma_{23} \\ \sqrt{2} \sigma_{13} \\ \sqrt{2} \sigma_{12} \\
\end{pmatrix} =
\begin{pmatrix}
\sigma_{11} \\ \sigma_{22} \\ \sigma_{33} \\ \sqrt{2} \tau_{23} \\ \sqrt{2} \tau_{13} \\ \sqrt{2} \tau_{12} \\
\end{pmatrix}
\end{equation*}

Tenzor přetvoření
\begin{equation*}
\varepsilon_{ij} =
\begin{pmatrix}
\varepsilon_{11} & \varepsilon_{12} & \varepsilon_{13} \\ & \varepsilon_{22} & \varepsilon_{23} \\ \text{\small sym.} & & \varepsilon_{33} \\
\end{pmatrix} =
\begin{pmatrix}
\varepsilon_{11} & \frac{\gamma_{12}}{2} & \frac{\gamma_{13}}{2} \\ & \varepsilon_{22} & \frac{\gamma_{23}}{2} \\ \text{\small sym.} & & \varepsilon_{33} \\
\end{pmatrix}
\quad\rightarrow\quad
\varepsilon_{a} =
\begin{pmatrix}
\varepsilon_{11} \\ \varepsilon_{22} \\ \varepsilon_{33} \\ \sqrt{2} \varepsilon_{23} \\ \sqrt{2} \varepsilon_{13} \\ \sqrt{2} \varepsilon_{12} \\
\end{pmatrix} =
\begin{pmatrix}
\varepsilon_{11} \\ \varepsilon_{22} \\ \varepsilon_{33} \\ \frac{\sqrt{2}}{2} \gamma_{23} \\ \frac{\sqrt{2}}{2} \gamma_{13} \\ \frac{\sqrt{2}}{2} \gamma_{12} \\
\end{pmatrix}
\end{equation*}

Tenzor tuhosti
\begin{equation*}
C_{ijkl}
\quad\rightarrow\quad
C_{ab} =
\begin{pmatrix}
C_{1111} & C_{1122} & C_{1133} & \sqrt{2} C_{1123} & \sqrt{2} C_{1113} & \sqrt{2} C_{1112} \\
& C_{2222} & C_{2233} & \sqrt{2} C_{2223} & \sqrt{2} C_{2213} & \sqrt{2} C_{2212} \\
&          & C_{3333} & \sqrt{2} C_{3323} & \sqrt{2} C_{3313} & \sqrt{2} C_{3312} \\
&          &          & 2 C_{2323} & 2 C_{2313} & 2 C_{2312} \\
& \text{\small sym.} &          &          & 2 C_{1313} & 2 C_{1112} \\
&          &          &          &          & 2 C_{1212} \\
\end{pmatrix}
\end{equation*}

Výhody:
\begin{itemize}
	\item zachování hustoty elastické energie
	\item zachování normy tenzorů
	\item napětí a~přetvoření je mapováno stejným způsobem
\end{itemize}

Nevýhody:
\begin{itemize}
	\item složky tenzoru tuhosti se nezachovávají
\end{itemize}
