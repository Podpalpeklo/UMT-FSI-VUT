% !TeX root = skripta-konstitutivni-vztahy.tex
% !TeX lastmodified = 2019-10-23

\subsection{Tenzor druhého Piolova-Kirchhoffova napětí}
Tenzor druhého Piolova-Kirchhoffova napětí je definován jako elementární síla $\diff F_{0i}$ vztažená na původní (tj. nedeformovanou) plochu elementu. Tato síla je však při přenášení na původní element změněna oproti skutečné síle $\diff F_i$ stejným poměrem jako elementární rozměr v~odpovídajícím směru. Ten se mění při zatížení podle vztahu
\begin{equation}
	\diff x_i = \frac{\partial x_i}{\partial X_i} \diff X_i,
\end{equation}
resp. při zpětné transformaci do nedeformovaného tvaru
\begin{equation}
	\diff X_i = \frac{\partial X_i}{\partial x_i} \diff x_i.
\end{equation}

Podobně transformujeme i~elementární sílu
\begin{equation}
	\diff F_{0i} = \frac{\partial X_i}{\partial x_i} \diff F_i,
\end{equation}
takže  napětí je
\begin{equation}
	S_i = \frac{\diff F_{0i}}{\diff X_j \diff X_k}
\end{equation}
Konkrétně pro normálové napětí ve směru 1 platí:
\begin{equation}
	S_i = \frac{\diff F_{01}}{\diff X_2 \diff X_3}
	= \frac{\frac{\partial X_1}{\partial x_1} \diff F_1}{\diff X_2 \diff X_3}
\end{equation}

Tento tenzor nemá jasný fyzikální význam, používá se proto, že je i~pro velká přetvoření symetrický a~je energeticky konjugovaný s~Green-Lagrangeovým tenzorem přetvoření.
