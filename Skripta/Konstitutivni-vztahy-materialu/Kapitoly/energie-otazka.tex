% !TeX root = skripta-konstitutivni-vztahy-materialu.tex
% !TeX lastmodified = 2019-04-01

\subsection{Vyhovuje hookovský materiál definici hyperelastických materiálů?}
Funkce měrné energie napjatosti byla v~lineární PP odvozena např. pro jednoosou napjatost ve tvaru
\begin{equation}
	W = \frac{1}{2} \sigma_1:\varepsilon_1 = \frac{1}{2} \bm{\sigma}:\bm{\varepsilon}
\end{equation}
a~platí
\begin{equation}\label{eq:saint-venant-stress}
	\bm{S} = \frac{\partial W}{\partial \bm{E}}.
\end{equation}

Podle rovnice (\ref{eq:saint-venant-stress}) pak derivace energie napjatosti podle přetvoření (pro malé deformace libovolně definovaného, např. smluvního přetvoření $\varepsilon$) opravdu určuje odpovídající složku napětí (pro malé deformace opět podle libovolné definice tenzoru napětí), jak ukazuje následující výraz:
\begin{equation}
	S_{ij} = \frac{\partial W}{\partial E_{ij}}
	= \frac{\partial W}{\partial \varepsilon}
	= \frac{\partial \left(\frac{1}{2} \sigma \varepsilon\right)}{\partial \varepsilon}
	= \frac{\partial \left(\frac{1}{2} E \varepsilon^2\right)}{\partial \varepsilon}
	= \frac{2}{2} E \varepsilon = \sigma
\end{equation}
\textbf{Lineárně elastický materiál je tedy jen zvláštním případem materiálu hyperelastického.}
Podobně lze vyjádřit napětí i~pro víceosou napjatost.

\subsubsection{Energie napjatosti hookovského materiálu pro víceosou napjatost}
Funkci měrné energie napjatosti odvozenou v~lineární PP pro jednoosou a~smykovou napjatost lze zobecnit do tvaru
\begin{equation}
	W = \frac{1}{2} \sigma_{ij} \varepsilon_{ij}.
\end{equation}

Při rozepsání tohoto vztahu do složek je třeba vzít v~úvahu, že podle \hyperref[sec:einsteinovo-scitaci-pravidlo]{Einsteinova pravidla} jsou oba indexy $i$ a~$j$ sčítací.
Parciální derivace této energie napjatosti podle kterékoli složky přetvoření určuje odpovídající složku napětí.
