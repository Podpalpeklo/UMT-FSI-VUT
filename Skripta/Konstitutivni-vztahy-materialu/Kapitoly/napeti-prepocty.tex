% !TeX root = skripta-konstitutivni-vztahy.tex
% !TeX lastmodified = 2010-03-16

\subsection{Vzájemné přepočtové vztahy pro tenzory napětí}
Nejvhodnější pro vzájemný přepočet tenzorů přetvoření jsou poměrná protažení $\lambda_i$, tedy složky tenzoru deformačního gradientu. V~hlavním souřadnicovém systému platí následující vztahy:

Hlavní Cauchyho (skutečné) napětí lze vyjádřit pomocí 1.P.K. napětí
\begin{equation}
	\sigma_i = \frac{\diff F_i}{\diff x_j \diff x_k}
	= \frac{\diff F_i}{\lambda_j \diff X_j \lambda_k \diff X_k}
	= \frac{\tau_i}{\lambda_j \lambda_k}
\end{equation}

Pro nestlačitelný materiál platí $\lambda_i\lambda_j\lambda_k=1$ a~tedy napětí jsou ve vztahu
\begin{equation}
	\sigma_i = \frac{\tau_i}{\lambda_j \lambda_k} = \tau_i \lambda_i
\end{equation}

Podobně lze vyjádřit 2.P.K. napětí
\begin{equation}
	S_i = \frac{\diff F_{0i}}{\diff X_j \diff X_k}
	= \frac{\frac{\partial X_i}{\partial x_i} \diff F_i}{\diff X_j \diff X_k}
	= \frac{\frac{1}{\lambda_i} \diff F_i}{\diff X_j \diff X_k}
	= \frac{1}{\lambda_i} \tau_i
\end{equation}

Cauchyho napětí lze vyjádřit i~pomocí 2.P.K. napětí
\begin{equation}
	\sigma_i = \frac{\tau_i}{\lambda_j \lambda_k}
	= \frac{\lambda_i}{\lambda_j \lambda_k} S_i,
\end{equation}
nebo jednodušeji pro nestlačitelný materiál
\begin{equation}
	\sigma_i = \lambda_i \tau_i = \lambda_i^2 S_i.
\end{equation}
