% !TeX root = skripta-konstitutivni-vztahy.tex
% !TeX lastmodified = 2018-09-25

\subsection{Co jsou konstitutivní modely?}
Už v~PPI a~PPII byly uvedeny základní (lineární) konstitutivní vztahy pro modelování závislostí mezi deformací a napětím.
Tyto závislosti jsou dány vlastnostmi materiálu, které byly vytvořeny (konstituovány) přírodou, stvořitelem.
V~širším smyslu se k~nim v~mechanice kontinua počítají závislosti mezi dalšími veličinami odvozenými z~napětí a~přetvoření, souvisejícími se závislostí na čase, např. rychlostí deformace.
Jejich matematický popis, ať už lineární nebo nelineární, musí být nutně do jisté míry zjednodušený, proto se pro něj používá označení konstitutivní vztahy nebo konstitutivní modely.

\framebox[\linewidth]{Konstitutivní model je tedy matematický, příp. grafický popis konstitutivní závislosti.}

Konstitutivní závislosti (v mechanice) jsou příčinné závislosti mezi tenzory napětí a~přetvoření, příp. s~nimi matematicky souvisejícími veličinami, s uvažováním časových závislostí.

\subsection{Příklady konstitutivních modelů}

S~jakými konstitutivními modely jsme se již setkali? 
\begin{itemize}
	\item Ideálně tuhý materiál
	\item Lineárně elastický materiál (izotropní nebo anizotropní -- vláknové kompozity) -- Hookův zákon
	\item Elasticko-plastický materiál (ideálně, tj. bez zpevnění, anebo se zpevněním)
	\item Tuho-plastický materiál
\end{itemize}

Jaké další modely budou probírány v předmětu „Konstitutivní vztahy materiálu“?
\begin{itemize}
	\item Viskoelastické modely materiálu (napětí a~deformace jsou mj. funkcí času) s~malými deformacemi i~s~velkými deformacemi (visko-hyperelastické)
	\item Hyperelastické modely materiálu  (materiál vykazující velká elastická přetvoření) izotropní a~anizotropní
	\item Elasticko-plastické modely materiálu
	\item Modely porušení
\end{itemize}

\subsection{Existují konstitutivní modely tekutin?}
\begin{itemize}
	\item Ideální kapalina
	\item Newtonská kapalina
	\item Nenewtonské kapaliny -- mezi smykovým napětím a~rychlostí deformace platí jiné závislosti než proporcionální
	\item Ideální plyn
	\item Viskózní plyn
\end{itemize}

Tyto modely předpokládají jistou formu závislosti mezi napětími a~rychlostí tvarových změn (přetvoření), proto je lze také zahrnout mezi konstitutivní modely.

\subsection{Rozdělení konstitutivních modelů}
Je užitečné zavést následující rozdělení kostitutivních modelů podle jejich složitosti: 
\begin{itemize}
	\item Základní konstitutivní modely -- sem řadíme nejjednodušší kostitutvní modely, definující ideální látku v jejích základních skupenstvích -- ideální tuhou látku, ideální kapalinu a~ideální plyn.
	\item Jednoduché konstitutivní modely -- Pod tímto pojmem jsou chápány ty konstitutivní modely, které popisují chování látek, jež se od  „ideálních“ odlišují určitou jedinou vlastností, např. ideálně elastická látka, ideálně plastická látka, viskózní (newtonská) kapalina.
	\item Kombinované konstitutivní modely – Jsou to modely, které vzniknou kombinací dvou nebo vícero jednoduchých konstitutivních modelů. Využívají mj. tzv. reologických modelů k~popisu chování např. těchto druhů látek: viskoelastických, elasticko-plastických, viskoplastických a~elasticko-viskoplastických.
\end{itemize}

Konstitutivních modelů existuje velké množství.
Aspoň částečný přehled o~nich, včetně vzájemných souvislostí, poskytuje schéma konstitutivních modelů.

\subsection{Charakteristiky základních konstitutivních modelů}
\begin{itemize}
	\item Ideálně tuhá látka
	\begin{itemize}
		\item nekonečně velký odpor proti deformaci
		\item deformace je pro daný problém nepodstatná
	\end{itemize}
	\item Ideální kapalina
	\begin{itemize}
		\item nulový odpor proti změně tvaru (nulová viskozita)
		\item nekonečně velký odpor proti změně objemu (nestlačitelnost)
	\end{itemize}
	\item Ideální plyn
	\begin{itemize}
		\item nulový odpor proti změně tvaru
		\item malý odpor proti změně objemu, daný stavovou rovnicí ideálního plynu
	\end{itemize}
\end{itemize}

Závěr: Pro vzájemné odlišení látek v~různých skupenstvích je nutné oddělovat objemovou a~tvarovou složku deformace.

\subsection{Rozlišování mezi objemovou a tvarovou složkou deformace}
U~konstitutivních modelů je obecně třeba oddělovat objemovou a~tvarovou složku deformace.

Tak tomu je např. i~u~plasticity -- podle podmínek plasticity je trvalá deformace dána pouze deviátorem tenzoru napětí, kulová část tenzoru napětí není schopna vyvolat trvalou deformaci.
Jediným konstitutivním vztahem, který toto rozdělení nemusí dodržovat, je Hookův zákon (díky platnosti principu superpozice pro lineární závislosti).

Tenzory přetvoření a+napětí je tedy třeba rozdělit takto:
\begin{itemize}
	\item Tenzor přetvoření = volumetrická (kulová) část + deviátor tenzoru přetvoření
	\item Tenzor napětí = kulový tenzor napětí + deviátor tenzoru napětí
\end{itemize}

\subsection{Typy závislostí vyjadřujících chování látek}
Chování látek z~pohledu mechaniky je komplexně popsáno těmito závislostmi: 
\begin{itemize}
	\item deformačně-napěťová odezva -- konstitutivní závislost v~užším slova smyslu -- je závislost mezi tenzorem napětí a~tenzorem přetvoření (nutnost použití tenzorového, resp. maticového počtu). Z~praktických důvodů se často používají zjednodušené tvary tohoto vyjádření, platné pouze pro specifické případy napjatosti (jednoosá, dvojosá rovnoměrná, smyková, trojosá rovnoměrná).
	\item Creepová odezva -- Je to závislost deformace na čase, označovaná jako tečení (creep). Obvykle se vyšetřuje při statickém zatížení vyvolávajícím jednoosou napjatost danou napětím:
	\begin{equation}
		\sigma = \sigma_0 H(t),
		\quad\text{případně}\quad
		\sigma = \sigma_0 H(t) - \sigma_0 H(t-t_0),
	\end{equation}
	kde $H(t)$ je Heavisideova funkce, pro níž platí:
	\begin{equation}
		H = 0 \quad\text{pro}\quad t < 0,
		H = 1 \quad\text{pro}\quad t \geq 0.
	\end{equation}
	\item Relaxační odezva -- Je to závislost napětí na čase. Relaxační odezva se obvykle vyšetřuje při stavu deformace daném přetvořením:
	\begin{equation}
		\varepsilon = \varepsilon_0 H(t),
		\quad\text{resp.}\quad
		\varepsilon = \varepsilon_0 H(t) - \varepsilon_0 H(t - t_0),
	\end{equation}
	kde $H(t)$ je stejná Heavisideova funkce.
	\item Rychlostní odezva -- U~látek, které vykazují časovou závislost deformačně-napěťové odezvy na zatížení se uvádí i~závislost napětí $\sigma$ na rychlosti deformace $\dot{\varepsilon}$ (tedy $\sigma-\dot{\varepsilon}$).
\end{itemize}




