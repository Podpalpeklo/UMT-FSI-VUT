% !TeX root = skripta-konstitutivni-vztahy.tex
% !TeX lastmodified = 2010-03-16

\subsection{Modifikované invarianty Cauchy-Greenova tenzoru deformace}
Modifikované (též někdy označované jako redukované) invarianty Cauchy-Greenova tenzoru deformace se používají pro popis deviátorové složky deformace. Tak jako deviátor tenzoru malých přetvoření vznikl odečtením středního přetvoření od jednotlivých složek délkových přetvoření, zde dostaneme příslušná poměrná protažení dělením jednotlivých složek středním protažením $\lambda_s$. 

Pak dostaneme modifikovaná hlavní protažení z~rovnice
\begin{equation}
	\bar{\lambda}_p = J^{-\frac{1}{3}} \lambda_p \quad (p = 1,2,3)
\end{equation}

Modifikované invarianty Cauchy-Greenova tenzoru deformace jsou pak dány 
\begin{align}
	\bar{I}_1
	&= \bar{\lambda}_1^2 + \bar{\lambda}_2^2 + \bar{\lambda}_3^2
	= \left(\lambda_1^2 + \lambda_2^2 + \lambda_3^2\right) J^{-\frac{2}{3}}
	= I_1 J^{-\frac{2}{3}}
	= I_1 I_3^{-\frac{1}{3}}\\
	\bar{I}_2
	&= \bar{\lambda}_1^2 \bar{\lambda}_2^2 + \bar{\lambda}_2^2 \bar{\lambda}_3^2 + \bar{\lambda}_1^2 \bar{\lambda}_3^2
	= \left(\lambda_1^2 \lambda_2^2 + \lambda_2^2 \lambda_3^2 + \lambda_1^2 \lambda_3^2\right) J^{-\frac{4}{3}}
	= I_2 J^{-\frac{4}{3}}
	= I_2 I_3^{-\frac{2}{3}}
\end{align}
