% !TeX root = skripta-konstitutivni-vztahy.tex
% !TeX lastmodified = 2016-10-10

\subsection{Cauchyho (logaritmický) tenzor přetvoření}
Nedostatky tenzorů konečných přetvoření:
\begin{itemize}
	\item Green-Lagrange: změny délek jsou stále vztahovány k~původním hodnotám, zatímco aktuální již mohou být v průběhu procesu zatěžování významně odlišné.
	\item Almansi-Hamel: změny délek jsou stále vztahovány ke konečným (deformovaným) hodnotám délek, zatímco aktuální od nich mohou být v~průběhu procesu zatěžování ještě podstatně odlišné.
\end{itemize}

Cauchyho definice přetvoření je exaktnější v~tom, že infinitezimální přírůstek délky vztahuje vždy k~aktuální délce v~daném stádiu zatěžovacího procesu.

Přetvoření úsečky o~původní délce $X_{i0}$, která se vlivem zatížení mění na aktuální hodnotu $X_i$ až dosáhne konečné (deformované) délky $x_{ik}$, určíme integrací přírůstků její délky $\diff x_i$:
\begin{equation}
	E^C_{ij}
	= \int\limits_{X_{i0}}^{x_{ik}} \frac{1}{x_i} \diff x_i
	= \left. \ln(x) \right\rvert_{X_{i0}}^{x_{ik}}
	= \ln(x_{ik}) - \ln(X_{i0})
	= \ln\left(\frac{x_{ik}}{X_{i0}}\right)
	= \ln(\lambda_i)
\end{equation}

\subsubsection{Tenzorová formulace Cauchyho tenzoru přetvoření}

Složky tohoto tenzoru jsou rovny přirozeným logaritmům odpovídajících složek tenzoru deformačního gradientu.


Souřadnice tohoto tenzoru jsou tedy rovny přirozeným logaritmům odpovídajících souřadnic tenzoru deformačního gradientu.
\begin{equation}
	E^C_{ij} = \ln(F_{ij}) = \frac{1}{2} \ln(C_{ij})
\end{equation}
