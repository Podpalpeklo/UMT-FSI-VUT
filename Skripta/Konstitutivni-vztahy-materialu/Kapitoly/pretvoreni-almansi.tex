% !TeX root = skripta-konstitutivni-vztahy.tex
% !TeX lastmodified = 2019-10-23

\subsection{Tenzor Almansiho-Hamelova přetvoření}
Podle Almansiho se poměrné přetvoření vztahuje ke konečným (nedeformovaným) rozměrům. Pak délkové přetvoření lze vyjádřit (označení viz obrázek \ref{fig:pretvoreni}):
\begin{equation}
	E^A_x = \frac{A'B' - AB}{A'B'}
\end{equation}

Podobným postupem jako pro tenzor Greenova-Lagrangeova přetvoření se dospěje k obecnému tenzorovému zápisu ve tvaru:
\begin{equation}
	E^A_{ij}
	= \frac{1}{2} \left[ \frac{\partial u_i}{\partial x_j} + \frac{\partial u_j}{\partial x_i} - \frac{\partial u_k}{\partial x_j} \frac{\partial u_k}{\partial x_i} \right]
\end{equation}
Také tento zápis používá Einsteinovo sčítací pravidlo.

Praktické použití tohoto tenzoru je omezeno tím, že konečné (deformované) souřadnice obvykle předem neznáme.
