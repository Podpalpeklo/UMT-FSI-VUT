% !TeX root = skripta-konstitutivni-vztahy.tex
% !TeX lastmodified = 2017-09-19

\subsection{Newtonův zákon viskozity}
vyjadřuje lineární závislost mezi gradientem rychlosti a~smykovým napětím v~tekutině.
Obvykle se uvádí ve tvaru:
\begin{equation}
	\bm{\tau} = \eta\, \nabla \bm{c},
\end{equation}
nebo též rozepsané na složky:
\begin{equation}
	\tau_{ij} = \eta \left(\begin{matrix}
		\frac{\partial c_1}{\partial X_1} & \frac{\partial c_1}{\partial X_2} & \frac{\partial c_1}{\partial X_3}\\
		\frac{\partial c_2}{\partial X_1} & \frac{\partial c_2}{\partial X_2} & \frac{\partial c_2}{\partial X_3}\\
		\frac{\partial c_3}{\partial X_1} & \frac{\partial c_3}{\partial X_2} & \frac{\partial c_3}{\partial X_3}
	\end{matrix}\right)
\end{equation}
kde
\begin{description}
	\item[{$\tau\:[\si{\pascal}]$}] je smykové napětí,
	\item[{$\eta\:[\si{\pascal\second}]$}] je tzv. dynamická viskozita,
	\item[{$\nabla \bm{c}\:[\si{\per\second}]$}] je gradient rychlosti tekutiny.
\end{description}

Kapaliny, jejichž chování lze popsat Newtonovým zákonem viskozity označujeme jako newtonské.
Viskozita je jejich materiálovou charakteristikou, která je ovšem teplotně závislá.
