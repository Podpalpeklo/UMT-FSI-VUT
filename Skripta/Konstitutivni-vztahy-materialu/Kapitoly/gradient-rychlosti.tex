% !TeX root = skripta-konstitutivni-vztahy-materialu.tex
% !TeX lastmodified = 2019-04-01

\subsection{Jaký je vztah mezi gradientem rychlosti a~úhlovou deformací?}
Tvar Newtonova zákona viskozity známý z~hydromechaniky lze modifikovat následovně:
\begin{equation}
	\tau
	= \eta \frac{\diff c}{\diff n}
	= \eta \frac{\partial c_x}{\partial y}
	= \eta \frac{\partial \left(\frac{\partial u_x}{\partial t}\right)}{\partial y}
	= \eta \frac{\partial \left(\frac{\partial u_x}{\partial y}\right)}{\partial t}
	= \eta \frac{\partial \gamma}{\partial t}
	= \eta \dot{\gamma},
\end{equation}
kde
\begin{description}
	\item[{$\tau\:[\si{\pascal}]$}] je smykové napětí,
	\item[{$\eta\:[\si{\pascal\second}]$}] je tzv. dynamická viskozita,
	\item[{$\tfrac{\diff c}{\diff n}\:[\si{\per\second}]$}] je gradient rychlosti kapaliny.
\end{description}
